\subsection{Zadanie 7}

\begin{enumerate}[label=(\alph*)]
\item
$$
P(X=0) = e^{-2,4}*\frac{2,4^{0}}{0!}=e^{-2,4}
$$

$$
P(X=1)= e^{-2,4}*\frac{2,4^{1}}{1!}=e^{-2,4}*2,4
$$
$$ 
EX=2,4
$$
$$
Var(X)=2,4
$$

\item Żeby policzyć prawdopodobieństwo, że zmienna losowa będzie
miała wartość powyżej 5 można policzyć prawdopodobieństwo zdarzenia
przeciwnego.

$$
P(X>5)=1-P(X=0)-P(X=1)-P(X=2)-P(X=3)-P(X=4)-P(X=5)
$$

$$
P(X>5)= e^{-2,4}
-e^{-2,4}*2,4 
-\frac{e^{-2,4}*2,4^{2}}{2}- 
\frac{e^{-2,4}*2,4^{3}}{6}-
\frac{e^{-2,4}*2,4^{4}}{24}-
\frac{e^{-2,4}*2,4^{5}}{120}
$$

Zmienna losowa o rozkładzie Poissona opisuje zmienne, w których zdarzenia występują rzadko.
\end{enumerate}
