\subsection{Zadanie 15}

Ze zbioru
\[
 \{1, 2, 3,... ,29, 30\}
\]
 losujemy 10 różnych liczb. Znaleźć prawdopodobieństwa następujących zdarzeń:
 \\
\\a) zdarzenie A = wszystkie wylosowane liczby są nieparzyste
\\
\\ b) zdarzenie B = dokładnie 3 liczby podzielne są przez 5,
\\
\\ c) zdarzenie C = wylosowano 5 liczb parzystych, 5 liczb nieparzystych, w tym dokładnie jedna liczba podzielna przez 10?

\[
n=30,k=10
\]

\[
|\Omega|={n \choose k}={30 \choose 10} = 30045015
\]

\[
|A|={15 \choose 10} = \frac{15!}{5!\cdot10!} = \frac{10 !\cdot 11 \cdot 12 \cdot 13 \cdot 14 \cdot 15}{10! \cdot 125} = 3003
\]

\[
|B|={6 \choose 3} * {24 \choose 7} = \frac{3! \cdot 4 \cdot 5 \cdot 6}{3! \cdot 6} \cdot \frac{17!\cdot18\cdot19\cdot20\cdot21\cdot22\cdot23\cdot24}{17!\cdot7!}= 20 \cdot 346104 = 6 922 080 
\]

\[|C|={15 \choose 5} * {12 \choose 4} * {3 \choose 1} = \frac{10! \cdot 11 \cdot 12 \cdot 13 \cdot 14 \cdot 15}{10! \cdot 5!}  \cdot \frac{8!\cdot9\cdot10\cdot11\cdot12}{8!\cdot4!} \cdot 3= 3003 \cdot 495 \cdot 3 = 29 700\]

Obliczamy prawdopodobieństwa
\[
P(A)=\frac{|A|}{|\Omega|}=\frac{3003}{30045015}=\frac{1}{10005}.
\]

\[
P(B)=\frac{|B|}{|\Omega|}=\frac{ 6 922 080 }{30045015}=\frac{14752}{10 015 005}.
\]
\[
P(C)=\frac{|C|}{|\Omega|}=\frac{ 29 700}{30045015}=\frac{60}{60697}.
\]

