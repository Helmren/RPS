\subsection{Zadanie 5}

Zakładamy, że talia podana w zadaniu jest talią pokerową i zawiera
po cztery karty 9, 10, J, Q, K, A.

Prawdopodobieństwo wyciągnięcia dziesiątki z jednej talii ($|A|=4$,
gdyż w talii są cztery 10-tki, $|\Omega|=24$, gdyż losujemy tylko jedną z 24 kart):

$$
P(A)=\frac{|A|}{|\Omega|}=\frac{4}{24}=\frac{1}{6}
$$

Prawdopodobieństwo sukcesu, czyli wyciągnięcia cztery razy dziesiątki
z czterech talii:
$$
P(B)=P(A)^4=(\frac{1}{6})^4=\frac{1}{1296}\approx0,077\%
$$

